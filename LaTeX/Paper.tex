\documentclass[11pt,a4paper]{vutinfth}
\usepackage[utf8]{inputenc}
\usepackage{graphicx}
\usepackage{textcomp}
\usepackage{caption}
\usepackage{mathtools}
\usepackage{amssymb}
\usepackage{algorithm}
\usepackage{algpseudocode}
%\usepackage{wrapfig}



%TODO: CHANGE!
\setauthor{}{Jakob Klinger}{}{male}
\setadvisor{Ass.Prof. Dipl.-Inform. Dr.rer.nat}{Martin N\''{o}llenburg}{}{male}%{Pretitle}{Forename Surname}{Posttitle}{male}
\setfirstassistant{University Assistant, M.Sc., B.Sc. }{Fabian Klute}{}{male}
%Ass.Prof. Dipl.-Inform. Dr.rer.nat. Martin Nöllenburg
% University Assistant, M.Sc., B.Sc. Fabian Klute

\setaddress{TODO: ADDRESS!}%TODO
\setregnumber{1125755}
\setdate{25}{09}{2017}

\settitle{Boundary Labeling for annotated documents}{Boundary Labeling in annotierten Dokumenten}
\setsubtitle{}{}%{TestSubtitleENG}{TestSubtitelGER}

\setthesis{bachelor}
\setcurriculum{Bachelor's programme Software \& Information Engineering}{Bachelorstudium Software \& Information Engineering} %TODO!

\begin{document}

\frontmatter
%\addtitlepage{naustrian}
\addtitlepage{english}
\addstatementpage

%\begin{danksagung*}
%\todo{Ihr Text hier.}
%\end{danksagung*}

%\begin{acknowledgements*}
%\todo{Enter your text here.}
%\end{acknowledgements*}

%\begin{kurzfassung}
%\todo{Ihr Text hier.}
%\end{kurzfassung}

%\begin{abstract}
%\todo{Enter your text here.}
%\end{abstract}

\selectlanguage{english}

\tableofcontents

\mainmatter


%TODO: Abstract: \begin{abstract}...\end{abstract}

\chapter{Introduction}%Wort ``Path'' durch nicht-definierte Terminologie ersetzen?
Annotating a document is usually solved by adding footnotes in an appropriate position and adding a simple reference in the text, leaving the reader to find the referenced content by themselves. If a more obvious connection between the text and the referenced content is required, the reference placed to the side of the document and visibly connected to the text by drawing a straight line or a more complex path between them.

In this paper, we will look at ways to use Boundary Labeling, which means that all annotations will be placed somewhere outside of the text they are referencing and will be visually connected to the feature they are referencing. (See also \cite{Bekos2007}) 

The guidelines on how to create suitable labelings are as follows: the connections should be as direct as possible, no important information should be obscured, and it should be easily discernable which Label belongs where. These three criteria easily come into conflict with one another, as the text usually is very dense and leaves little space for lines in between, yet one shouldn't allow them to pass through the text, as this makes the text harder to read. 

While there are many papers discussing Boundary Labeling in general, only very few exist that apply this concept to written text. Generally, this approach isn't used very often, and tends to use simplistic algorithms which produce mediocre results. %Beispiele von suboptimalen Lösungen einfügen? (Bilder)
However, the papers that do discuss boundary labeling in text offer interesting contributions.

In the paper about the Luatodonotes-Package\cite{Kindermann2014} illustrates some of the different styles of drawing these connecting paths, and came to the conclusion that paths without bends are easier to follow. %Bild einfügen?
However, most solutions proposed in that paper do not consider whether a path overlaps with text or not, which results in a decrease in readability.

The paper by Loose\cite{Loose2015} on the other hand is based around only using the free space between lines and words, which produces longer paths, and forces curves, but doesn't obscure any part of the text.
%Siehe auhc andere Papers - Motivation! (Warum will ich mir dieses Thema ansehen)


%TODO: Bezug auf Programm & Ergebnisse nehmen!

\section{Terminology and Fundamentals} %Evtl in Introduction eingliedern?
While Boundary Labeling (or an equivalent concept) can be applied to a space with different geometry or more dimensions, this paper will only concern itself with two-dimensonal, euclidean space.
To easily reference important concepts, some additional terminology will be introduced as well. (See Fig.~\ref{fig:term} for a visual explanation)%Add references to all terminology pictures!

%Add figure for graph terminology.

\begin{figure}%{r}{0.35\textwidth}
 \captionsetup{justification=centering, margin=0.75cm}
 \centering
  \includegraphics[scale=0.5]{IPE_TerminologyDrawing.pdf}
  \caption{Illustrated guide to the labeling terminology}
 \label{fig:term}
\end{figure}

A \emph{graph} $G=\langle V, E \rangle$ is a tuple of \emph{vertices} $V=\{v_1, v_2, ..., v_n\}$ and \emph{edges} $E=\{e_1, e_2, ..., e_m\}$. A vertex $v$ is a featureless object. %Wie  sage ich, dass die Umsetzung der Vertices sehr flexibel ist?
 Each edge $e$ is a relation between two vertices $E:V\rightarrow  V\times V$. %Edges can also be directional, or have a weight, which affects how they are treated by algorithms.
We call two vertices $u,v \in V$ \emph{adjacent}, if the edge $e=(u,v) \in E$.
 A \emph{path} $P=v_1, ..., v_h$ is an ordered series of vertices, where each vertex must have an edge connecting it to the subesquent one.
 \emph{Depth-first search} is a searching algorithm on a graph G that starts at a given vertex $v \in V$ and explores the graph by traversing its edges as far as possible before backtracking. In our algorithm, the goal is to reach a target vertex $v_t \in V$, at which point the algorithm terminates and returns the route taken from $v_s$ to $v_t$.
% A search algorithm is \emph{monotonous} if, for each vertex $v$ and each successor $v'$ generated by the algorithm, the estimated cost of reaching the goal from $v$ is not greater than the cost of reaching $v'$ plus the estimated cost of reaching the goal from $v'$.

\emph{polylines} are a connected series of \emph{Line segments}. Line segments are straight lines that contain each point between their starting and end point. %Open vs. closed line segments?
\emph{labels} hold additional information and are represented as boxes containing this information. They are usually placed in the \emph{Label area} which is an area designated to hold labels and is located off to a side so the labels don't obscure anything.
The point or object that this information refers to is called a \emph{site}. The site and the label are connected via a \emph{leader}, a polyline that can be further classified by looking at the orientation of its segments: \emph{O-Segments} run orthogonally to the border of the label area. \emph{P-Segments} run parallel to the border of the label area, and as such must be combined with other segments for the leader to reach its destination. \emph{S-Segments} are not required to have any particular orientation, and simply connect their start and einding points in a straight line.
The leader's name is created by combining the name of the segments - for example, the leader from Fig~\ref{fig:term} would be classified as an OPO-Leader.
The location where a leader connects to the label is called the \emph{port}. It can be restricted to pre-defined positions. %TODO: Re-insert monotonicity

% \textbf{Monotonous/Monotonicity:} Describes a steady progression towards a direction or goal. Will be used in this paper to indicate that there's no loss of progress made from the site to the label, effectively restricting the direction a leader can go at each bend.
%+Node für Knoten in Polylinien?

\chapter{The Algorithm}

In our algorithm, the focus was put on keeping the text as readable as possible. This means that leaders aren't allowed to pass through words, which was implemented by enclosing each word in a bounding rectangle that leaders aren't allowed to pass through.
The leaders should also be kept as short as possible, so we restricted them to always be moving toward the label, even if the direct route is unavailable.%Wie drücke ich das als Formel aus?
We also wanted to use the space available in the labeling area as efficiently as possible, so labels are placed as far up as possible in order to maximize the space available to future labels at the cost of increased leader length.


\begin{figure}%{r}{0.35\textwidth}
 \captionsetup{justification=centering, margin=0.75cm}
 \centering
  \includegraphics[scale=1.0]{RoutingGraphPartial_NoBorders.png}
  \caption{Visual representation of the routing graph. The highlighted nodes represent leader sources.}
 \label{fig:RGraph}
\end{figure}
To create leaders that exclusively use the space not taken up by a word's bounding rectangle, we decided to use a graph similar to the one Loose \cite{Loose2015} used.
As each vertex represents a physical location, they will have co-ordinates associated with them, and the vertices representing the sites will hold additional information regarding its leader and label. 
The graph is constructed by placing vertices between the lines, located next to each corner of a word's bounding rectangle, with two consecutive words in a line sharing the two vertices associated with their adjacent corners. %Insert picture
For the sites, we inserted an additional vertex above the center of the word, which will serve as the leader's starting point. After placing all vertices, we created edges between each vertex and the closest horizontal neighbour to both sides, and between nodes that are located exactly above or below each other, and exactly one line apart. (For a visual representation, see Fig.~\ref{fig:RGraph})

In our labeling algorithm, we decided to work through the labels in the order they appear in the text, placing each as far up as possible, and skipping any label that would've required us place the label below its leader's source node. We also opted to use fixed ports located in the top left corner of each label, as this allowed us to unabiguously place annotations by only knowing their port location, which is equal to their leader's ending point. To not restrict the label's placement by the line spacing, we left a buffer zone between the text area and the label area, which allows us to place labels even further above to optimally use the available space.
The algorithm's input consists of the Graph $G$ and a set of sites $V_{ann} \subset V$ which will be routed using a depth-first search algorithm, restricted to only selecting vertices located above or to the right of the current vertex, prioritizing moving up whenever possible. After the algorithm terminates, it returns a set with routing information for each vertex, which contains a Path $P=\{v_1, \cdots, v_n\}$, leading from the site to the text area's border, along with some additional information on how to draw the OPO-Segment that connects to the Label's port. If the routing for any given site failed, the path consists of a single vertex - the site.

\section{Implementation}
The program was written in Java 1.8.0u40, using JGraphT1.0.1\cite{JGraphT} as graph library. Since we only want to create leaders that don't intersect with the text, the graph was created alongside the placement of the words on the canvas.%, as it was easy to extract measurements at this point. Due to some problems with Java's various methods of calculating linebreaks, this process was done completely by hand.

%As the Graph's vertices represent fixed locations on the canvas, they each have coordinates associated with them, with the vertices representing the sites containing extra information, such as references to the corresponding annotation and to the leader connecting the two (if existent).
%The additional information stored in the vertices is used for both routing and drawing the leaders: The routing was done via a depth-first search algorithm that prioritized vertices located further up if possible, and vertices to the right otherwise. After the algorithm terminates, it returns  a vertex-based graph walk describing the route from the site to the border of the text area alongside information about where to draw the OPO-segment in the buffer zone if successful, or a walk containing only the starting vertex, if no valid path was found. Using this information alongside the positional data stored in the vertices, the polyline representing the leader can be drawn.
%TODO: Schönheitsfeatures, etc.
%TODO: Code genauer durchlesen und erweitern!

%\begin{algorithm}
%\caption{My algorithm}\label{euclid}
%\begin{algorithmic}[1]
%\Procedure{MyProcedure}{}
%\State $\textit{stringlen} \gets \text{length of }\textit{string}$
%\State $i \gets \textit{patlen}$
%\BState \emph{top}:
%\If {$i > \textit{stringlen}$} \Return false
%\EndIf
%\State $j \gets \textit{patlen}$
%\BState \emph{loop}:
%\If {$\textit{string}(i) = \textit{path}(j)$}
%\State $j \gets j-1$.
%\State $i \gets i-1$.
%\State \textbf{goto} \emph{loop}.
%\State \textbf{close};
%\EndIf
%\State $i \gets i+\max(\textit{delta}_1(\textit{string}(i)),\textit{delta}_2(j))$.
%\State \textbf{goto} \emph{top}.
%\EndProcedure
%\end{algorithmic}
%\end{algorithm}



% For demonstration purposes, we also included another algorithm that uses S-Leaders for this section.

%(Potential) Problems/Issues, solutions



%Section Ideas: The Program/Framework (Modellerklärung), The Algorithm(s), Implementation, Evaluation, Conclusion


\backmatter %TODO: Es gibt mehr nützliche Kommandos in VUTINFTH für zb Listen von Bildern, etc.!

\bibliographystyle{plain}
\bibliography{references}

\end{document}