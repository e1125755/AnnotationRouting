\documentclass[11pt,a4paper]{article}
\usepackage{termpaper}
\usepackage[utf8]{inputenc}
\usepackage{graphicx}
\usepackage{textcomp}
\usepackage{caption}

%opening
\title{<TITLE PENDING>}
\author{
 \authorname{Jakob Klinger} \\
 \studentnumber{1125755} \\
 \curriculum{033 534} \\
 \email{e1125755@student.tuwien.ac.at}
}

\begin{document}

\maketitle

%TODO: Abstract (\begin{abstract}...\end{abstract}

\section{Introduction}
Annotating a document is usually solved by adding footnotes or figures in an appropriate position and adding a simple reference in the text, leaving the reader to find the referenced content by themselves. Sometimes however, if a more obvious connection between the text and the referenced content is required, the reference is visibly connected to the text by drawing a line between them.

In this paper, we will look at ways to use Boundary Labeling, which means that all annotations will be placed outside of the text they are referencing. (See also \cite{Bekos2007}) While there are many papers discussing Boundary Labeling in general, only very few exist that apply this concept to written text. As a result, and since it is very easy to accidentally obscure parts of the text, this approach isn't used very often, and tends to use simplistic algorithms which produce mediocre results.

%TODO: Was genau wollen wir erreichen?
%TODO: Sektion: Terminologie


%Section Ideas: Leader Types, The Program/Framework, The Algorithm(s), 


\bibliographystyle{plain}
\bibliography{references}

\end{document}